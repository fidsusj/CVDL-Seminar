% This version of CVPR template is provided by Ming-Ming Cheng.
% Please leave an issue if you found a bug:
% https://github.com/MCG-NKU/CVPR_Template.

%\documentclass[review]{cvpr}
\documentclass[final]{cvpr}

\usepackage{times}
\usepackage{epsfig}
\usepackage{graphicx}
\usepackage{amsmath}
\usepackage{amssymb}

% Include other packages here, before hyperref.

% If you comment hyperref and then uncomment it, you should delete
% egpaper.aux before re-running latex.  (Or just hit 'q' on the first latex
% run, let it finish, and you should be clear).
\usepackage[pagebackref=true,breaklinks=true,colorlinks,bookmarks=false]{hyperref}


\def\cvprPaperID{****} % *** Enter the CVPR Paper ID here
\def\confYear{CVDL 2021}
\setcounter{page}{0} % For final version only


\begin{document}

%%%%%%%%% TITLE
\title{RepVGG: Making VGG-style ConvNets Great Again}

\author{Felix Hausberger\\
	Universität Heidelberg\\
	Grabengasse 1, 69117 Heidelberg\\
	{\tt\small eb260@stud.uni-heidelberg.de}
}

\maketitle

%%%%%%%%% ABSTRACT
\begin{abstract}
	
\end{abstract}

%%%%%%%%% BODY TEXT
\section{Introduction}

% Problem statement, motivation, main idea

%-------------------------------------------------------------------------
\section{Related Work}

% Fundamentals and related work

%-------------------------------------------------------------------------
\section{Approach}

% Explanatory figures

%-------------------------------------------------------------------------
\section{Experiments}

% Experiments and results

%-------------------------------------------------------------------------
\section{Discussion}

% Highlights and weaknesses

%-------------------------------------------------------------------------
\section{Conclusion}

% Own thoughts and conclusions

{\small
\bibliographystyle{ieee_fullname}
\bibliography{egbib}
}

\end{document}
